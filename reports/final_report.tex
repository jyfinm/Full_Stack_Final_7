% !TeX root = final_report.tex
\newcommand*{\PathToAssets}{../assets}%
\newcommand*{\PathToOutput}{../_output}%

%%%%%%%%%%%%%%%%%%%%%%%%%%%%%%%%%%%%%%
%% This file is compiled with XeLaTeX.
%%%%%%%%%%%%%%%%%%%%%%%%%%%%%%%%%%%%%%
\documentclass[12pt]{article}
\usepackage{my_article_header}
\usepackage{my_common_header}
\usepackage{graphicx}
\usepackage{float} % Allows using [H] to force figure placement

\begin{document}

\title{
Replicating the Nozawa Corporate Bond Portfolios from He, Kelly, and Manela (2017)
}
\author{
Jesse Yan \& Gabriel Ran
}
\maketitle

\begin{abstract}
This report documents our attempt to replicate the decile-sorted corporate bond portfolios originally constructed by Nozawa, as described in He, Kelly, and Manela (2017). We used the WRDS Bond Returns Database in combination with the latest WRDS (TRACE) Duration-Matched Treasury Returns (Open Source Bond Pricing, June 2024 release) contributed by Yoshio Nozawa to estimate yield spreads and replicate the subsequent returns. Notably, while the original HKM dataset covers 1970--2012, Nozawa's duration-matched Treasuries data is available from 2002 to 2023, limiting our replication to the overlapping period (2002--2012). Despite challenges such as missing yield data, negative amounts outstanding, and partial coverage of duration-matched Treasuries, our final decile returns closely track the original benchmark’s direction and ordering.
\end{abstract}

\section{Introduction}
This project aims to reproduce the corporate bond decile portfolios sorted by yield spreads, as introduced by Nozawa and detailed in He, Kelly, and Manela (2017). Our primary data sources are:
\begin{itemize}
  \item \textbf{WRDS Bond Returns Database} for corporate bond yield, return, and amount outstanding information, and
  \item \textbf{Latest WRDS (TRACE) Duration-Matched Treasury Returns} (Open Source Bond Pricing, June 2024 release) contributed by Yoshio Nozawa, which provides U.S. Treasury bond yields and returns aligned with the duration of each corporate bond.
\end{itemize}

While He, Kelly, and Manela (HKM) originally span 1970--2012, the duration‐matched Treasury dataset starts in 2002 and continues through 2023. Consequently, the only overlapping interval between the HKM dataset and Nozawa’s updated duration‐matched data is roughly 2002--2012, restricting our replication to that period.

\subsection{Data Coverage and Challenges}
We observed that no corporate bonds had a corresponding yield in the first available month (August 2002) to compare with the duration-matched Treasury yield, so we started our replication from September 2002 onward. In other months, missing yield data occurred because some bonds had no trading volume; we were unable to calculate yields for those bonds given the limited methodological guidance in the literature. Forward-filling the yields did not help; it in fact made it worse. Additionally, some bonds exhibited negative amounts outstanding, which required data cleaning. As a result, while the replicated portfolios follow the overall direction and trend of the benchmark from 2002 to 2012, exact numerical matching was not achieved.

\section{Methodology}
\subsection{Yield Spread Calculation}
To calculate the yield spread for each corporate bond, we subtracted the duration-matched YTM (yield to maturity) of the Treasuries (obtained from Open Source Bond Pricing) from the corporate bond yield (obtained from WRDS). We then formed deciles based on these yield spreads each month.

\subsection{Portfolio Construction and Returns}
After assigning each bond to a decile, we computed the subsequent month’s return for that bond. Each decile’s return was calculated as a weighted average of its constituent bonds’ returns, using the bond’s amount outstanding as the weight. Since no bonds had duration-matched Treasury yields in August 2002, the first month of return replication began in September 2002. The final sample ends in 2012 to match the last available year of the HKM data.

\section{Data Overview}
In this section, we show a snapshot of the replication sample and the benchmark data. Each table displays the first 10 and last 10 rows of the respective datasets.

\begin{table}[htbp]
\centering
\caption{Replication Portfolio Returns}
{\footnotesize
\renewcommand{\arraystretch}{0.9} % Adjust row spacing if needed
\resizebox{\textwidth}{!}{%
  \input{\PathToOutput/replication_table.tex}
}
}
\caption*{\footnotesize This table displays a head/tail snapshot of the replication portfolio returns, which are monthly from September 2002 onward. The yield spreads and returns are based on combining WRDS corporate bond data and duration-matched Treasury yields from Nozawa’s dataset.}
\label{table:replication_sample}
\end{table}

\begin{table}[htbp]
\centering
\caption{Benchmark US Corporate Bond Returns}
{\footnotesize
\renewcommand{\arraystretch}{0.9}
\resizebox{\textwidth}{!}{%
  \input{\PathToOutput/us_corp_table.tex}
}
}
\caption*{\footnotesize This table provides a similar head/tail snapshot of the benchmark returns from He, Kelly, and Manela (2017). It serves as the comparison point for evaluating our replicated portfolios, restricted to the overlapping 2002--2012 period.}
\label{table:benchmark}
\end{table}

\section{Summary Statistics}
In addition to the decile analysis, we computed summary statistics for both the benchmark and the replicated returns. The tables below present the mean, standard deviation, cumulative return, and the period over which the data are available (start and end dates) for each decile. Although the overall directional trends are similar between the benchmark and replication summaries, small differences in magnitude are evident. These discrepancies likely arise from missing yield data and other data cleaning challenges.

\begin{table}[htbp]
\centering
\caption{Benchmark Summary Statistics}
{\footnotesize
\renewcommand{\arraystretch}{0.9}
\resizebox{\textwidth}{!}{%
  \input{\PathToOutput/benchmark_summary.tex}
}
}
\caption*{\footnotesize This table provides summary statistics for the benchmark US corporate bond returns. The statistics indicate overall trends that align with the replicated data, albeit with slight differences in average returns and volatility.}
\label{table:benchmark_summary}
\end{table}

\begin{table}[htbp]
\centering
\caption{Replication Summary Statistics}
{\footnotesize
\renewcommand{\arraystretch}{0.9}
\resizebox{\textwidth}{!}{%
  \input{\PathToOutput/replicate_summary.tex}
}
}
\caption*{\footnotesize This table shows summary statistics for the replicated decile returns. Although these statistics capture the general direction of the benchmark data, differences in the mean, standard deviation, and cumulative return suggest that the replication is not a perfect match, likely due to data limitations.}
\label{table:replicate_summary}
\end{table}

\section{Replication Analysis}
Below is a table of the key replication metrics, treating each decile’s replicated returns relative to the benchmark. Although we achieved strong alignment in trends and directions, some differences emerged due to missing yield data and partial coverage of duration-matched Treasuries.

\begin{table}[htbp]
\centering
\caption{Decile Replication Analysis}
\resizebox{\textwidth}{!}{%
  \input{\PathToOutput/analysis_table.tex}
}
\caption*{\footnotesize Correlations, R$^2$, regression slope, intercept, MAE, RMSE, and tracking error for decile portfolios. High correlations and low errors suggest a solid replication, though exact agreement was not always possible.}
\label{table:analysis}
\end{table}

\section{Results and Discussion}
\subsection{Cumulative Returns of Decile Portfolios}
Figure~\ref{fig:cumulative_returns} shows the cumulative returns for each of the ten deciles. Although our replication focuses on the 2002--2012 overlap, we extended the plot for illustrative purposes using all available data points from Nozawa’s dataset. Notably, portfolios with the \textbf{lowest yield spreads} (e.g., Decile~11) tend to exhibit more stable, steady growth, reflecting a relatively lower risk profile. Conversely, portfolios with the \textbf{highest yield spreads} (e.g., Decile~20) show greater volatility—achieving higher peaks but also larger drawdowns, aligning with the higher-risk nature of high-spread bonds.

\begin{figure}[H]
\centering
\includegraphics[width=1\textwidth]{\PathToOutput/cumulative_returns.png}
\caption{Cumulative Returns of Replicated Decile Portfolios}
\caption*{\footnotesize This figure illustrates the cumulative returns for each yield-spread decile over time. Portfolios in lower deciles (lower spreads) show steadier returns and less volatility, while higher-spread deciles exhibit higher peaks and more pronounced drawdowns. The ordering confirms the risk-return relationship typically associated with yield spreads.}
\label{fig:cumulative_returns}
\end{figure}

\subsection{Overall Assessment}
Despite missing yield data and necessary data cleaning (e.g., handling negative amounts outstanding), we largely succeeded in replicating the decile returns for the overlapping 2002--2012 period. The correlation and regression statistics (Table~\ref{table:analysis}) indicate that our replicated returns closely track the benchmark. However, the summary statistics (Tables~\ref{table:benchmark_summary} and \ref{table:replicate_summary}) reveal small discrepancies in the mean, standard deviation, and cumulative returns between the benchmark and replicated data. These differences suggest that while the replication captures the overall directional trend and risk-return profile, exact numerical matching was not achieved due to data limitations.

\section{Conclusion}
Our replication of Nozawa’s corporate bond portfolios from He, Kelly, and Manela (2017) shows strong alignment in direction and decile ranking during the overlapping 2002--2012 window. The combination of WRDS corporate bond data and Nozawa’s duration-matched Treasury yields allowed us to construct yield spreads and subsequent returns starting in September 2002. Key challenges included missing yield data, negative amounts outstanding, and limited data coverage relative to the broader 1970--2012 HKM sample. Overall, however, the high correlations, low error measures, and similar decile trajectories indicate a largely successful replication effort, even though slight differences in summary statistics point to some inherent data challenges.

\end{document}